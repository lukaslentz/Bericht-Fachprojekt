%
\begin{simplebox}{Themen-Einleitung}
%
In der Einleitung soll die Motivation für die Beschäftigung mit der bearbeiteten Aufgabe geklärt werden.
Dazu sollen folgende Fragen adressiert werden:
%
\begin{itemize}
\item Worum geht es in der Arbeit? 
\item Was ist die Hauptfrage und warum ist diese interessant?
\item Gibt es bereits Arbeiten, die sich mit dieser oder einer ähnlichen Fragestellung beschäftigt haben?
\item Auf welchen Grundlagen kann aufgebaut werden?
\item Was ist das Ziel der Arbeit und worin besteht der Nutzen?
\item Welche Methode soll zur Beantwortung der Fragestellung verwendet werden?
\end{itemize}
%
%
\end{simplebox}
%
Im Rahmen des Studiums am Umweltcampus Birkenfeld ist sowohl in den Bachelor, als auch in den Master-Studiengängen die Bearbeitung eines Fachprojekts vorgesehen.
Der hierfür einzuplanende Arbeitsaufwand beträgt \SI{150}{\hour} und die Note und Leistungspunkte werden auf Grundlage eines schriftlichen Berichts und einer mündlichen Präsentation vergeben \cite{modulhandbuch}.
\par
Da die Anforderungen an wissenschaftliche Berichte sich in weiten Teilen von der Art und Weise, in welcher Alltagsliteratur wie z.B. Zeitungen, Sachbücher und Romane geschrieben sind unterscheiden, muss das wissenschaftliche Schreiben zumeist erst erlernt werden. 
Insbesondere in der frühen Studienphase ist davon auszugehen, dass die Studierenden hierbei Unterstüzung benötigen.
\par
Eine wichtige Frage die sich daraus für das Lehrpersonal an Hochschulen ergibt ist, wie die Studierenden beim erlernen des wissenschaftlichen Schreibens optimal unterstützt werden können.
Aus der großen Menge der bereits zur Verfügung stehenden Hilfsmittel wird der sehr fundierten und umfangreichen Ratgeber \cite{teHeesen} sowie die kompaktere und ebenfalls sehr gute Zusammenfassung \cite{Rumpler} empfohlen.
\par 
Ausgehend von diesen beiden Arbeiten soll in dem vorliegenden Dokument lediglich der erwartete Aufbau eines Projektberichts skizziert werden.
Hierdurch sollen zwei Ziele erreicht werden.
Einerseits soll eine Hilfestellung für die Anfertigung von wissenschaftlichen Arbeiten geben, und andererseits die spezifischen Bewertungskriterien von Dr.-Ing. Lentz transparent gemacht werden.
Um dies zu erreichen wurde der erwartete Aufbau eines Projektberichts im vorliegenden Dokument umgesetzt.  
Weiter findet sich zu Beginn eines jeden Abschnittes eine Auflistung der Punkte, die in diesem Abschnitt behandelt werden sollen.
Im Anschluss daran findet sich ein Beispieltext in dem gezeigt wird, wie die aufgelisteten Punkte umgesetzt werden können. 
%