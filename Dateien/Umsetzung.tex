%
\begin{simplebox}{Umsetzung}
%
Während es in den Abschnitten \ref{sec:Einleitung} und \ref{sec:Methodik} um die Planung der Arbeit ging, soll nun beschrieben werden, wie die konkrete Umsetzung erfolgte.
Ziel ist es eine Anleitung zu geben, die es entsprechend vorgebildeten Personen ermöglicht Ihre Arbeit zu wiederholen.
Dafür ist es notwendig, dass folgende Punkte adressiert werden:

%
\begin{itemize}
\item Auflistung der verwendeten Meßgeräte und Gerätschaften
\item Beschreibung der verwendeten Software, Programmiersprachen und Bibliotheken 
\item Kamm es bei der Umsetzung der geplannten Arbeitsschritte zu Schwierigkeiten
\item Wie wurden Probleme gelöst
\item Welche Bereiche waren besonders trickreich
\item Welche Produkte/Daten wurden erzeugt
\end{itemize}
%
\end{simplebox}
%
Für die Erstellung der Hilfestellung wurde das  plattformunabhängige und kostenfreie Softwarepaket \LaTeX verwendet.
Nach einer gewissen Einarbeitungszeit lassen sich hiermit sehr systematisch und robust Dokumente mit ansprechendem Layout erzeugen, wobei eine besondere Stärke in der Darstellung von Formeln liegt.
Bei Interesse finden sich auf der Homepage des Verfassers zahlreiche Links zu Minimalbeispielen, die den Einstieg in die Verwendung von \LaTeX\,erleichtern.
Die Dateien welche zur Erstellung der Hilfestellung verwendet wurden, können Sie unter \url{https://github.com/lukaslentz/Bericht-Fachprojekt} ebenfalls herunterladen und weiterverwenden.
Obwohl \LaTeX\, also viele Vorzüge für das wissenschftliche Schreiben bietet, sind Sie natürlich frei in der Wahl des Textverarbeitungssystems, welches Sie zum Schreiben des Berichts verwenden wollen. 
%