%
%
\tikzset{
arrow/.style = {-latex,thick},
task/.style = {draw,align=center,minimum height=1.5cm,minimum width=5.5cm,inner sep=2pt, outer sep=0pt},
data/.style = {draw,align=center,trapezium, trapezium left angle=70, trapezium right angle=110,minimum height=1.2cm,inner sep=10pt, outer sep=5pt}
}
%
\begin{figure}[h!]
\centering
\begin{tikzpicture}[node distance=1.2cm]
%
\node[task,rounded corners](start){Start};
\node[task,below = of start.center](t1){Klärung der\\ Aufgabenstellung};
\node[task, below = of t1.center] (t2) {Literaturrecherche\\Einarbeitung in das Thema};
\node[task, below = of t2.center] (t3) {Planung der Methodik \\ zur Bearbeitung der Aufgabe};
\node[task, below = of t3.center] (t4) {Umsetzung \\ der geplanten Methodik};
\node[task, below = of t4.center] (t5) {Auswertung und\\ Darstellung der Ergebnisse};
\node[task, below = of t5.center] (t6) {Fazit};
%
\node[task,minimum width=4cm, left = of t1] (l1) {Absprache \\mit Betreuer};
\node[data, below = of l1.center](l2){Fachliteratur};
\node[task,minimum width=4cm, below = of l2.center] (l3) {Absprache \\mit Betreuer};
\node[data,below = of l3.center](l4){Daten};
%
\node[data, right = of t2](r1){Einleitung};
\node[data, below = of r1.center](r2){Methodik};
\node[data, below = of r2.center](r3){Umsetzung};
\node[data, below = of r3.center](r4){Ergebnisse};
\node[data, below = of r4.center](r5){Fazit};
%
\draw[arrow](start)--(t1);
\draw[arrow](t1)--(t2);
\draw[arrow](t2)--(t3);
\draw[arrow](t3)--(t4);
\draw[arrow](t4)--(t5);
\draw[arrow](t5)--(t6);
%
\draw[latex-latex,thick](l1)--(t1);
\draw[arrow](l2)--(t2);
\draw[latex-latex,thick](l3)--(t3);
\draw[arrow](t4)--(l4);
\draw[arrow](l4)|-(t5);
%
\draw[arrow](t2)--(r1);
\draw[arrow](t3)--(r2);
\draw[arrow](t4)--(r3);
\draw[arrow](t5)--(r4);
\draw[arrow](t6)--(r5);
%
\end{tikzpicture}
\caption{Flussdiagramm zur Verdeutlichung des Ablaufs der Bearbeitung des Fachprojekts.}
\label{fig:flow}
\end{figure}
%
%